\documentclass{article}
%%%%%%%%%%%%%%%%%%%%%%%%%%%
% paquetes
%%%%%%%%%%%%%%%%%%%%%%%%%%%
\usepackage[utf8]{inputenc}
\usepackage[spanish]{babel}
\usepackage{graphicx}
\usepackage{caption}
\usepackage{amsmath}
\usepackage{xcolor}
\usepackage{url}
\usepackage{hyperref}
\hypersetup{
    colorlinks=true,
    linkcolor=blue,
    filecolor=magenta,
    urlcolor=cyan,
    pdftitle={Overleaf Example},
    pdfpagemode=FullScreen,
    }
\urlstyle{same}
\usepackage{listingsutf8} % Paquete para soportar caracteres UTF-8 en listings
\lstset{
    inputencoding=utf8, % Permite caracteres especiales
    extendedchars=true, % Activa caracteres extendidos
    literate=%
        {á}{{\'a}}1 {é}{{\'e}}1 {í}{{\'i}}1 {ó}{{\'o}}1 {ú}{{\'u}}1
        {Á}{{\'A}}1 {É}{{\'E}}1 {Í}{{\'I}}1 {Ó}{{\'O}}1 {Ú}{{\'U}}1
        {ñ}{{\~n}}1 {Ñ}{{\~N}}1,
    language=[LaTeX]TeX,
    breaklines=true,
    basicstyle=\ttfamily\scriptsize,
    keywordstyle=\color{blue},
    identifierstyle=\color{magenta},
}

\spanishdatedel
\usepackage[left=3cm, right=3cm, top=2.5cm, bottom=2.5cm]{geometry}
%%%%%%%%%%%%%%%%%%%%%%%%%%%

%%%%%%%%%%%%%%%%%%%%%%%%%%%
\title{Título del documento}
\author{Luis Robles}
\date{\today}
%%%%%%%%%%%%%%%%%%%%%%%%%%%
\begin{document}
\maketitle

\begin{abstract}
Irure ut duis fugiat adipisicing anim sunt id occaecat culpa ea. Lorem tempor voluptate aliquip commodo occaecat non veniam. Ex magna voluptate incididunt eu exercitation enim reprehenderit est eu do eiusmod anim.
% falta mejorar el resumen, traducir
\end{abstract}

\section{Introducción a LaTeX}
En el mundo de la creación de documentos, especialmente en el ámbito académico y científico, es fundamental contar con herramientas que no solo permitan escribir contenido, sino también garanticen una presentación clara, profesional y consistente. A lo largo de los años, muchos investigadores, profesores y estudiantes han buscado formas de mejorar la calidad tipográfica de sus trabajos y simplificar la gestión de bibliografías, citas, tablas, figuras, y ecuaciones matemáticas complejas.

En este contexto, surge \LaTeX{}, una herramienta poderosa y ampliamente utilizada para la preparación de documentos de alta calidad. A diferencia de los procesadores de texto tradicionales, donde el formato y el contenido se gestionan de manera visual e inmediata, \LaTeX{} ofrece un enfoque diferente, permitiendo a los autores concentrarse en el contenido y la estructura de su documento, mientras que el formato es manejado por el sistema. Este enfoque no solo facilita la creación de documentos complejos sino que también asegura que el resultado final cumpla con los más altos estándares de presentación tipográfica.

\subsection{¿Qué es LaTeX?}
\LaTeX{}\footnote{Desarrollado por Leslie Lamport} es un sistema de preparación de documentos de alta calidad, especialmente adecuado para la escritura de documentos científicos y técnicos que contienen fórmulas matemáticas complejas. Es una colección de macros y comandos que extienden la funcionalidad de \TeX\footnote{Desarrollado por Donald Knuth en la década de 1970}.

\LaTeX{} se centra en la separación del contenido y el formato, permitiendo a los usuarios escribir en un formato sencillo y claro, mientras que el sistema se encarga de la presentación visual. Esto hace que sea especialmente útil para documentos largos y complejos, donde la consistencia y la calidad tipográfica son importantes.

\subsection{Instalación de \LaTeX{} y herramientas necesarias}
\begin{center}
  \begin{Large}
    \textbf{Ver video} \\
  \end{Large}
  \url{https://drive.google.com/drive/folders/1vUHs7PPenpT4D2z5t3HZPozbD5wcjbM6?usp=sharing}
\end{center}
\clearpage
\subsection{Primer documento en LaTeX}

\subsubsection{Estructura básica de un documento en \LaTeX}
\begin{lstlisting}[caption={Contenido mínimo de un documento en \LaTeX}]
\documentclass{article}
% preámbulo
% sección de paquetes

\begin{document}
    Contenido del documento
\end{document}
\end{lstlisting}

\section{Formato y Estructuración de Documentos}

\subsection{Secciones y subsecciones}
\subsection{Formato de texto (negritas, cursivas, subrayados, etc.)}

\subsubsection{Negritas}

\subsubsection{Cursivas}
\subsubsection{Subrayados}


\subsection{Items y Listas}
\subsubsection{Entorno itemize}
el torno itemize permite listar el contenido en base a \textbf{ítems}

\begin{minipage}{0.5\textwidth}
  \begin{lstlisting}[language = TeX, caption = Codigo emjemplo]
    \begin{itemize}
      \item Primer item
      \item Segundo item
      \item Tercer item
    \end{itemize}
    \end{lstlisting}
\end{minipage}
\begin{minipage}{0.5\textwidth}
  \begin{itemize}
    \item Primer item
    \item Segundo item
    \item Tercer item
    \end{itemize}
\end{minipage}

\subsubsection{Entorno enumerate}
Permite listar el contenido mediante números

\begin{minipage}{0.5\textwidth}
  \begin{lstlisting}[language = TeX, caption = Codigo emjemplo]
    \begin{enumerate}
      \item Primer item enumerado
      \item Segundo item enumerado
      \item Tercer item enumerado
    \end{enumerate}
    \end{lstlisting}
\end{minipage}
\begin{minipage}{0.5\textwidth}
  \begin{enumerate}
    \item Primer item enumerado
    \item Segundo item enumerado
    \item Tercer item enumerado
  \end{enumerate}
\end{minipage}

%%%%%%%%%%%%%%%%%%%%%%%%%%%%%%%%%%%%%%%%%%%%%%%%%
\subsection{Figuras}
Para insertar una figura podemos usar el comando \textbf{ \textbackslash includegraphics\{ruta de la figura\}}

\begin{minipage}{0.7\textwidth}
  \begin{lstlisting}[caption = Ejemplo para incertar figura]
    \includegraphics[width = 0.3\textwidth]{img/logoMFP.png}
    \end{lstlisting}
\end{minipage}
\begin{minipage}{0.5\textwidth}
  \includegraphics[width = 0.3\textwidth]{img/logoMFP.png}
\end{minipage}

También se puede usar el entono \textbf{figure}

\begin{minipage}{0.5\textwidth}
  \begin{lstlisting}[caption = entorno figure]
    \begin{figure}[h]
      \centering
      \includegraphics[scale=0.3]{img/logoMFP.png}
      \caption{Logo de MFP}
      \label{fig:logoMFPef}
    \end{figure}
    \end{lstlisting}
\end{minipage}
\begin{minipage}{0.5\textwidth}
  \centering
    \includegraphics[scale=0.3]{img/logoMFP.png}
    \captionof{figure}{Logo de MFP}
    \label{fig:logoMFPef}
\end{minipage}

\subsubsection{argumentos extras}
\begin{itemize}
  \item scale
  \item angle
  \item width (en \textbf{cm})
  \item height (en \textbf{cm})
  \item clip
  \item trim
\end{itemize}

\subsection{Tablas}

\begin{itemize}
  \item Revisar \url{https://www.tablesgenerator.com/}
\end{itemize}


\subsection{Manejo de múltiples archivos y \texttt{\textbackslash input}}


\section{Matemáticas y Símbolos Científicos}
Para insertar ecuaciones como parte de un texto se emplea el símbolo dolar (\$) antes y después de la ecuación por ejemplo: $\vec{F} = m \vec{a}$

\subsection{Símbolos matemáticos y científicos}
Revisar: \url{https://metodos.fam.cie.uva.es/~latex/apuntes/apuntes3.pdf}

\subsection{Entorno displaymath}
% $$\vec{F} = m \vec{a}$$


\begin{displaymath}
  \mu = e^{\int P \, dx}
\end{displaymath}

% Integral en el exponente de e con límites
\begin{displaymath}
  e^{\displaystyle\int_{i}^{n} \frac{1}{cd} \, dx}
\end{displaymath}




\subsection{Entorno equation}
\begin{equation}
  \mu=e^{\displaystyle \int P dx}
\end{equation}

\subsection{Matrices y alineación de ecuaciones}

\begin{minipage}{0.5\textwidth}
  \begin{lstlisting}[caption = Matrix 01]
    \begin{equation}
      \begin{matrix}
      a & b\\
      c & d
      \end{matrix}
      \end{equation}
    \end{lstlisting}
\end{minipage}
\begin{minipage}{0.5\textwidth}
  \begin{equation}
    \begin{matrix}
    a & b\\
    c & d
    \end{matrix}
    \end{equation}
\end{minipage}

\begin{minipage}{0.5\textwidth}
  \begin{lstlisting}[caption = Matrix 01]
    \begin{equation}
      \begin{pmatrix}
      2 & 5 & 0\\
      7 & 3 & 8\\
      3 & 0 & 1
      \end{pmatrix}
      \end{equation}
    \end{lstlisting}
\end{minipage}
\begin{minipage}{0.5\textwidth}
  \begin{equation}
    \begin{pmatrix}
    2 & 5 & 0\\
    7 & 3 & 8\\
    3 & 0 & 1
    \end{pmatrix}
    \end{equation}
\end{minipage}



\textbf{Ver más ejemplos en}: \url{https://manualdelatex.com/tutoriales/matrices}

\begin{minipage}{0.5\textwidth}
  \begin{lstlisting}[caption = Ejemplo para alinear ecuaciones]
    \begin{align}
      2S & = (1+n)+(2+n-1)+\dots+(n+1),\\
         & = n(n+1).
      \end{align}
    \end{lstlisting}
\end{minipage}
\begin{minipage}{0.5\textwidth}
  \begin{align}
    2S & = (1+n)+(2+n-1)+\dots+(n+1),\\
       & = n(n+1).
    \end{align}
\end{minipage}

\begin{equation}
  \begin{split}
    H_c&=\frac{1}{2n} \sum^n_{l=0}(-1)^{l}(n-{l})^{p-2}
    \sum_{l _1+\dots+ l _p=l}\prod^p_{i=1} \binom{n_i}{l _i}\\
    &\quad\cdot[(n-l )-(n_i-l _i)]^{n_i-l _i}\cdot
    \Bigl[(n-l )^2-\sum^p_{j=1}(n_i-l _i)^2\Bigr].
  \end{split}
\end{equation}

\textbf{Revisar más ejemplos en}: \\
\url{https://matematicas.uam.es/~fernando.chamizo/asignaturas/2122latex/guides/guia06.pdf}

\section{Gestión de Referencias, Bibliografías y Normativas}
\begin{enumerate}
    \item Introducción a BibTeX% y Biber
    \item Introducción a Zotero y Mendely
    \item Creación y gestión de bibliografías
    \item Gestión de citas con \texttt{biblatex}% o \textit{natbib}
    \item Estilos de citas (APA, IEEE, etc.)
    \item Formato APA en LaTeX
    \item Configuración de citas y referencias según APA
    \item Otras normativas comunes (Chicago, MLA, etc.)
    % \item Integración de diferentes normativas en un mismo documento
\end{enumerate}

\section{Documentos Académicos y Científicos}
\begin{enumerate}
    \item Redacción de artículos científicos
    \item Producción de libros y reportes largos
    \item Plantillas y clases de documentos específicas
    \item Plantilla de la Revista de investigación de Física (RIF) de la UNMSM
    \item Formatos de Tesis y proyectos de investigación
\end{enumerate}







\end{document}

