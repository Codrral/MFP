\subsection{Símbolos matemáticos y científicos}
Revisar: \url{https://metodos.fam.cie.uva.es/~latex/apuntes/apuntes3.pdf}

\subsection{Entorno displaymath}
% $$\vec{F} = m \vec{a}$$
\begin{displaymath}
  \mu = e^{\int P \, dx}
\end{displaymath}

% Integral en el exponente de e con límites
\begin{displaymath}
  e^{\displaystyle\int_{i}^{n} \frac{1}{cd} \, dx}
\end{displaymath}

\subsection{Entorno equation}
\begin{equation}
  \vec{F} = m \vec{a}
  \label{ec:2new}
\end{equation}

\subsection{Matrices y alineación de ecuaciones}

\begin{equation}
  \begin{matrix}
    a & b\\
    c & d
  \end{matrix}
\end{equation}


\begin{minipage}{0.5\textwidth}
  \begin{lstlisting}[caption = Matrix 01]
    \begin{equation}
      \begin{matrix}
      a & b\\
      c & d
      \end{matrix}
      \end{equation}
    \end{lstlisting}
\end{minipage}
\begin{minipage}{0.5\textwidth}
  \begin{equation}
    \begin{matrix}
    a & b\\
    c & d
    \end{matrix}
    \end{equation}
\end{minipage}

\begin{minipage}{0.5\textwidth}
  \begin{lstlisting}[caption = Matrix 01]
    \begin{equation}
      \begin{pmatrix}
      2 & 5 & 0\\
      7 & 3 & 8\\
      3 & 0 & 1
      \end{pmatrix}
      \end{equation}
    \end{lstlisting}
\end{minipage}
\begin{minipage}{0.5\textwidth}
  \begin{equation}
    \begin{pmatrix}
    2 & 5 & 0\\
    7 & 3 & 8\\
    3 & 0 & 1
    \end{pmatrix}
    \end{equation}
\end{minipage}

\textbf{Ver más ejemplos en}: \url{https://manualdelatex.com/tutoriales/matrices}

\begin{minipage}{0.5\textwidth}
  \begin{lstlisting}[caption = Ejemplo para alinear ecuaciones]
    \begin{align}
      2S & = (1+n)+(2+n-1)+\dots+(n+1),\\
         & = n(n+1).
      \end{align}
    \end{lstlisting}
\end{minipage}
\begin{minipage}{0.5\textwidth}
  \begin{align}
    2S & = (1+n)+(2+n-1)+\dots+(n+1),\\
       & = n(n+1).
    \end{align}
\end{minipage}

\begin{equation}
  \begin{split}
    H_c&=\frac{1}{2n} \sum^n_{l=0}(-1)^{l}(n-{l})^{p-2}
    \sum_{l _1+\dots+ l _p=l}\prod^p_{i=1} \binom{n_i}{l _i}\\
    &\quad\cdot[(n-l )-(n_i-l _i)]^{n_i-l _i}\cdot
    \Bigl[(n-l )^2-\sum^p_{j=1}(n_i-l _i)^2\Bigr].
  \end{split}
\end{equation}


\begin{equation}
\label{eq:suma}
\sum^n_{i=1} f (x) = e^{\displaystyle \int}
\end{equation}

